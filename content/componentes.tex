Antes de se fazer uma análise do que levou o jogo \textit{Flappy Bird} a ter tanto sucesso, primeiro devem-se introduzir seus componentes.

\subsection{Arte}
A arte do jogo é extremamente simples, sendo desenvolvida em oito \textit{bits} e com pequenas variações randômicas a cada turno: a cor do pássaro e o céu também, dando ideia de dia ou noite. Apesar de os efeitos serem simples, são também agradáveis. Os obstáculos do jogo representados por canos, bem como outros elementos do cenário, seguem o padrão dos jogos da franquia do \textit{Super Mario}. Outro ponto de simplicidade na arte do jogo é o fato de a paisagem de fundo ser totalmente estática, o que se tornou extremamente raro num contexto em que o efeito de rolamento \textit{parallax} se tornou tão difundido \cite{Eldic2014}.

\subsection{Sons}
Os sons do jogo se limitam àqueles que representam seus evento: impulsionar o pássaro, passar por um obstaculo e morrer. Com isso, o jogo não apresenta nenhuma música de fundo, se mostrando bastante simples também neste aspecto \cite{Eldic2014}.

