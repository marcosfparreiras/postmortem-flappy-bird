Antes de se fazer uma análise do que levou o jogo \textit{Flappy Bird} a ter tanto sucesso, primeiro devem-se introduzir seus componentes, que tiveram influência determinante nele.

\subsection{Arte}
A arte do jogo é extremamente simples, sendo desenvolvida em oito \textit{bits} e com pequenas variações randômicas: a cor do pássaro e do céu mudam a cada turno, dando ideia de múltiplos personagens e de dia ou noite, respectivamente. Apesar de os efeitos serem simples, são também agradáveis. Os obstáculos do jogo representados por canos, bem como outros elementos do cenário, seguem o padrão dos jogos da franquia do \textit{Super Mario}. Outro ponto de simplicidade na arte do jogo é o fato de a paisagem de fundo ser totalmente estática, o que se tornou extremamente raro num contexto em que o efeito de rolamento \textit{parallax} se tornou tão difundido \cite{Eldic2014}.

\subsection{Sons}
Os sons do jogo se limitam àqueles que representam seus eventos: impulsionar o pássaro, passar por um obstaculo e morrer. Com isso, o jogo não apresenta nenhuma música de fundo, se mostrando bastante simples também neste aspecto \cite{Eldic2014}.

\subsection{\textit{Game Design}}
Os mecanismos do \textit{Flappy Bird} são extremamente simples, provados e fortes, sendo ele um \textit{jogo de helicóptero} em que, a partir de um toque, o objeto controlado sobe e, quando não se toma nenhuma ação, ele cai devido à ação da gravidade \cite{Eldic2014}.

O \textit{Fappy Bird}, entretanto, incorpora algumas características que tornam este jogo ainda mais interessante: a primeira delas é o fato de não se poder segurar o dedo apertado, fazendo com que seja sempre necessário apertar a tela novamente para evitar que o pássaro caia. Isso aumenta o nível de dificuldade do jogo, o tornando mais desafiador. Uma segunda característica é o fato de o jogo não ter sua dificuldade alterada em nenhum momento, com velocidade e tamanho dos obstáculos sempre constantes \cite{Eldic2014}.

\subsection{\textit{Marketing}}
Nenhuma campanha de \textit{marketing} foi feita partindo de Nguyen. Sua divulgação foi totalmente baseada no sucesso em que se tornou, a partir de cada vez mais menções nas redes sociais como \textit{Facebook} e \textit{Twitter} \cite{Eldic2014}.

\subsection{Monetização}
A monetizãção do \textit{Flappy Bird} foi baseada nos anúncios exibidos durante o jogo. Com isso, a distribuição do jogo foi gratuira, durante todo o tempo em que esteve disponível. O sucesso do jogo fez com que as arrecadações com anúncios ultrapassassem os cinqenta mil dólares por dia \cite{Eldic2014}.