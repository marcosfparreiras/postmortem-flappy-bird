A origem do jogo \textit{Flappy Bird} data de novembro de 2012, quando o vietnamita Dong Nguyen divulgou em seu \textit{Twitter} uma primeira imagem divulgando o início da criação de mais um jogo \cite{Warren2014}. Nguyen era o único funcionário do estúdio de jogos .Gear, fundado por ele pŕopio \cite{Harvard2014}. Após a fase de refinamento, o jogo, originalmente nomeado \textit{Flappy Flappy} foi renomeado para \textit{Flappy Bird} e lançado na \textit{App Store} em abril de 2013. A alteração no nome se deveu ao fato de, na época, já existir um jogo na loja com o nome \textit{Flappy Flappy}.

O jogo se manteve no anonimato até outubro de 2014, quando ainda ocupava a posição número 1469 dentre os aplicativos da categoria Família na \textit{App Store}. Nos mês de novembro, o jogo começou a se popularizar a partir de divulgação de usuários no \textit{Twitter} e no \textit{Facebook}, fazendo com que, no início de dezembro, já ocupasse a posição 74 na categoria Família, e sendo o aplicativo número 1308 nos Estados Unidos \cite{Warren2014}.

Devido ao seu extremo nível de dificuldade, usuários de redes sociais passaram a postar publicamente seus casos de amor e ódio com o jogo, elevando sua visibilidade ainda mais e fazendo com que no dia dez de janeiro de 2014, alcançasse o \textit{top 10} da \textit{App Store} nos EUA. Neste mesmo mês, no dia vinte e dois, foi lançada na \textit{Google Play} a versão do jogo para \textit{Android} e em menos de uma semana ele alcançou o posto de aplicativo mais baixado da loja \cite{Warren2014}.

Os mais de cinquenta milhões de downloads do jogo fizeram com que a mídia focasse bastante em seu criador, Dong Nguyen, que chegou inclusive a ser acusado de usar técnicas ilegais para aumentar o número de \textit{reviews} do seu jogo. Segundo as acusações, ele estaria utilizando um robô para gerar \textit{reviews} automaticamente para, desta forma, promover seu jogo mais rapidamente.Estas acusações entretanto nunca foram confirmadas \cite{Warren2014}.

Outra prova do sucesso do jogo foi o fato de alcançar o primeiro lugar na \textit{App Store} em 53 países diferentes, mostrando que seu sucesso foi a nível global \cite{Warren2014}.

Neste momento, o jogo estava gerando cerca de U\$\$ 50.000,00 por dia com anúncios e Nguyen já começara a sucumbir a toda a pressão sobre ele e sobre seu jogo. Com isto, no dia oito de fevereiro de 2014, ele postou uma mensagem oficial no seu \textit{Twitter} informando aos usuários do jogo que ele seria tirado do ar no dia seguinte. Muitos acreditram que seria uma questão de \textit{marketing} ou que ele tivera algum problema legal relacionado ao jogo. Ambas as suposições estavam erradas: Nguyen tirou o jogo da loja porque, segundo ele, as pessoas estavam deixando de viver suas vidas para jogar o jogo que ele tinha criado \cite{Warren2014}.

Desta forma, no dia nove de fevereiro, vinte e oito dias após entrar no \textit{top 10}, o jogo foi de fato retirado do ar, acumulando mais de cinquenta milhões de \textit{downloads} e mais de dezesseis milhões de \textit{tweets} relacionados \cite{Warren2014}.


