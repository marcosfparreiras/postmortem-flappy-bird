Apesar da grande simplicidade do jogo, ele alcançou um sucesso estrondoso. Os principais motivos disso são:

% \subsection*{É simples, mas muito difícil}
% % \item[É simples, mas muito difícil] 
% Todo o funcionamento do jogo é baseado num único controle: o clique na tela. Com isso, não é necessária nem mesmo uma tela de instruções permitindo que qualquer pessoa que tenha o jogo em mãos, possa jogá-lo. O contraponto à simplicidade do jogo surge no seu nível de dificuldade extremamente elevado, fazendo com que o jogo seja fácil de se aprender, mas difícil de se dominar. A simplicidade faz com que os jogadores se cobrem na obtenção de melhores resultados, que são difíceis de se conquistar. Isso dá ao jogo o tom desafiador e viciante \cite{Dino2014}.
% % \end{itemize}

\begin{itemize}
\item É simples, mas muito difícil: Todo o funcionamento do jogo é baseado num único controle: o clique na tela. Com isso, não é necessária nem mesmo uma tela de instruções permitindo que qualquer pessoa que tenha o jogo em mãos possa jogá-lo. O contraponto à simplicidade do jogo surge no seu nível de dificuldade extremamente elevado, fazendo com que o jogo seja fácil de se aprender, mas difícil de se dominar. A simplicidade faz com que os jogadores se cobrem na obtenção de melhores resultados, que são difíceis de se conquistar. Isso dá ao jogo o tom desafiador e viciante \cite{Dino2014}.
\item Se baseia na nostalgia: A aparência do jogo remete muito à franquia \textit{Super Mario}, fazendo com que uma simpatia quase instantânea seja estabelecida pelos fãs da famosa franquia quando jogam o \textit{Flappy Bird} \cite{Dino2014}.
\item Não possui atalhos: Ao contrário de diversos jogos atuais que oferecem facilidades a partir de compras dentro dos aplicativos, como dobrar pontuação e reduzir dificuldade por exemplo, o \textit{Flappy Bird} não oferece nenhuma circunstância desigual. Com isso, a pontução obitda pelo jogador é exatamente a merecida por ele, acirrando disputas entre amigos e tornando o jogo um fenômeno social \cite{Dino2014}.
% \end{itemize}
\end{itemize}

% \begin{itemize}
% \item[É simples, mas muito difícil]: Todo o funcionamento do jogo é baseado num único controle: o clique na tela. Com isso, não é necessária nem mesmo uma tela de instruções permitindo que qualquer pessoa que tenha o jogo em mãos, possa jogá-lo. O contraponto à simplicidade do jogo surge no seu nível de dificuldade extremamente elevado, fazendo com que o jogo seja fácil de se aprender, mas difícil de se dominar. A simplicidade faz com que os jogadores se cobrem na obtenção de melhores resultados, que são difíceis de se conquistar. Isso dá ao jogo o tom desafiador e viciante \cite{Dino2014}.

% \item[Se baseia na nostalgia]: A aparência do jogo remete muito à franquia \textit{Super Mario}, fazendo com que uma simpatia quase instantânea seja compreendida pelos fãs da famosa franquia quando jogam o \textit{Flappy Bird} \cite{Dino2014}.

% \item[Não possui atalhos]: Ao contrário de diversos jogos atuais que oferecem facilidades a partir de compras dentro dos aplicativos, como dobrar pontuação e reduzir dificuldade por exemplo, o \textit{Flappy Bird} não oferece nenhuma circunstância desigual. Com isso, a pontução obitda pelo jogador é exatamente a merecida por ele, acirrando disputas entre amigos e tornando o jogo um fenômeno social \cite{Dino2014}.
% \end{itemize}





